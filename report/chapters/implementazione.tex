\chapter{Implementazione}
\label{ch:implementazione} % This how you label a chapter and the key (e.g., ch:into) will be used to refer this chapter ``Introduction'' later in the report. 
% the key ``ch:into'' can be used with command \ref{ch:intor} to refere this Chapter.
parla di decomposizione delle funzionalità in microservizi
dipendenze minime tra i servizi
reattività agli eventi
\section{Tecnologie}
(swagger, postman, autogenerazione javadoc?, rabbitMQ, JWT per autenticazione utenti,
 docker, nodejs, typescript, mongodb, jest, angular, figma, ...)
\subsection{Swagger - OpenAPI}

Uno dei requisiti fondamentali di questo progetto è stata la documentazione delle API.
Per fare ciò è stato utilizzato Swagger, un framework open-source che permette di descrivere, produrre e consumare servizi web RESTful. 
Swagger permette di testare le API direttamente dalla documentazione, grazie a un'interfaccia grafica che consente di inviare richieste e visualizzare le risposte.

\vspace{1cm}

Per i servizi creati in Node.js è stata utilizzata una libreria ad hoc, in grado di generare la documentazione automaticamente tramite i corretti decoratori.
Al contrario, per i servizi in Java non esiste alcuna libreria in grado di generare automaticamente la documentazione API a partire dai metodi esposti, quindi è stata creata manualmente.
Grazie a un lavoro preliminare open-source svolto da \href{https://github.com/anupsaund}{\underline{Anup Saund}} sulla sua repository \href{https://github.com/anupsaund/vertx-auto-swagger}{\underline{Vertx auto swagger}}, è stato possibile creare automaticamente un file HTML contenente la documentazione delle API esposte dai servizi in Java. 
Purtroppo, questo file non era in grado di aggiornarsi automaticamente in caso di modifiche alle rotte, quindi è stato necessario adattarlo per renderlo più flessibile e adatto al nostro progetto.

\vspace{1cm}

Vertx è una libreria davvero completa e ben strutturata. È stato possibile creare un Router in grado di gestire tutte le rotte del progetto in Java utilizzando una semplice classe astratta che ha definito tutti i metodi utilizzati nel progetto, tramite la quale si è potuto razionalizzare il codice e creare dinamicamente la funzionalità richiesta.

\begin{lstlisting}[language=Java, caption={Semplice interfaccia per le rotte HTTP}, label=list:java_swagger_interface]
package httpRest;

import io.vertx.core.Handler;
import io.vertx.core.http.HttpMethod;
import io.vertx.ext.web.RoutingContext;

public interface IRouteResponse {
    HttpMethod getMethod();
    String getRoute();
    Handler<RoutingContext> getHandler();
}
\end{lstlisting}

La parte realmente complicata è stata fare in modo che le annotazioni presenti nel codice sorgente venissero lette e trasformate in un file JSON che rappresentasse correttamente la documentazione delle API. 
Dopo alcuni tentativi, è stato possibile automatizzare anche questa operazione.

\begin{lstlisting}[language=Java, caption={Aggiunta dei moduli delle classi contenenti rotte HTTP}, label=list:java_swagger_modules]
final ImmutableSet<ClassPath.ClassInfo> modelClasses = ImmutableSet.<ClassPath.ClassInfo>builder()
        .addAll(this.getClassesInPackage("game"))
        .addAll(this.getClassesInPackage("userModule"))
        .addAll(this.getClassesInPackage("BLManagement"))
        .addAll(this.getClassesInPackage("chatModule"))
        .build();
\end{lstlisting}

\vspace{1cm}

Java permette di creare oggetti molto complessi, diversi dai JSON / object solitamente utilizzati in JavaScript. Era fondamentale poter adoperare questi schemi per avere una documentazione ricca e per poter testare velocemente le API senza utilizzare necessariamente un client come Postman. 

Infine, per poter mostrare correttamente la documentazione, è stato necessario creare un file HTML che permettesse di visualizzare la documentazione in modo chiaro e ordinato. Questo file statico, presente nella repository del progetto, utilizza come fonte il file JSON di OpenAPI che viene rigenerato a ogni build del progetto, in modo da avere sempre la documentazione aggiornata.

\subsection{Postman}


\subsection{Log automatici}

Il servizio core del sistema è il middleware e per poterlo monitorare è stato necessario implementare un sistema di log automatici. Tenere traccia di tutte le operazioni è un compito molto complesso e avrebbe reso difficile un eventuale debug su un software in produzione quindi 
grazie a Vertx, che offre gratuitamente un logger e alla libreria \href{https://mvnrepository.com/artifact/ch.qos.logback/logback-classic}{\underline{LogBack}} i dei servizi più importanti sono trascritti su file .log in modo da poterli analizzare in caso di problemi.

Questo sistema poi si è evoluto grazie ad una configurazione di logback in un file xml tramite il quale è possibile personalizzare completamente il sistema di log, decidendo quali package loggare, in quali file trascriverli, scegliere addirittura il livello di log (info, debug, error, ecc...)
ed infine attuare un sistema di log rotate secondo il quale i log oltre una certa dimensione impostata da questa configurazione vengono compressi e rinominati in modo da non occupare troppo spazio ed è possible anche settare il numero di questi file compressi da mantenere.

\vspace{1cm}

Questo sistema si adatta perfettamente ad un software dockerizzabile che ha montato un volume relativo alla cartella /log, che può anche essere un volume remoto, da cui poter consultare velocemente i log in caso di problemi.


\section{Parte di distribuiti che non so dove mettere, sta nell'impl? p.s. dargli un nome decente}
FAULT DETECTION (con node nel cluster, nel nostro caso?)/TOLLERANCE
SCALABILITà?
verificare la scalabilità orizzontale (aumento richieste in un periodo)?
PERSISTENZA
CONSISTENZA
AFFIDABILITà
FAULTS (GIOCATORE SI DISCONNETTE NEL MEZZO DELLA PARTITA)
IL SISTEMA SI BLOCCA IN CASO DI ERRORI?
**GUARDA L'ESEMPIO DI POLITICHE AUTOVALUTAZIONE NELL'ANALISI**

monitoraggio per osservare stato attuale sistema (mi sa che noi non l'abbiamo)
\section{Screen}
\section{Optional(Mockup)}
% %%%%%%%%%%%%%%%%%%%%%%%%%%%%%%%%%%%%%%%%%%%%%%%%%%%%%%%%%%%%%%%%%%%%%%%%%%%%%%%%%%%
% \section{Problem statement}
% \label{sec:intro_prob_art}
% This section describes the investigated problem in detail. You can also have a separate chapter on ``Problem articulation.''  For some projects, you may have a section like ``Research question(s)'' or ``Research Hypothesis'' instead of a section on ``Problem statement.'

% %%%%%%%%%%%%%%%%%%%%%%%%%%%%%%%%%%%%%%%%%%%%%%%%%%%%%%%%%%%%%%%%%%%%%%%%%%%%%%%%%%%
% \section{Aims and objectives}
% \label{sec:intro_aims_obj}
% Describe the ``aims and objectives'' of your project. 

% \textbf{Aims:} The aims tell a read what you want/hope to achieve at the end of the project. The  aims define your intent/purpose in general terms.  

% \textbf{Objectives:} The objectives are a set of tasks you would perform in order to achieve the defined aims. The objective statements have to be specific and measurable through the results and outcome of the project.


% %%%%%%%%%%%%%%%%%%%%%%%%%%%%%%%%%%%%%%%%%%%%%%%%%%%%%%%%%%%%%%%%%%%%%%%%%%%%%%%%%%%
% \section{Solution approach}
% \label{sec:intro_sol} % label of Org section
% Briefly describe the solution approach and the methodology applied in solving the set aims and objectives.

% Depending on the project, you may like to alter the ``heading'' of this section. Check with you supervisor. Also, check what subsection or any other section that can be added in or removed from this template.

% \subsection{A subsection 1}
% \label{sec:intro_some_sub1}
% You may or may not need subsections here. Depending on your project's needs, add two or more subsection(s). A section takes at least two subsections. 

% \subsection{A subsection 2}
% \label{sec:intro_some_sub2}
% Depending on your project's needs, add more section(s) and subsection(s).

% \subsubsection{A subsection 1 of a subsection}
% \label{sec:intro_some_subsub1}
% The command \textbackslash subsubsection\{\} creates a paragraph heading in \LaTeX.

% \subsubsection{A subsection 2 of a subsection}
% \label{sec:intro_some_subsub2}
% Write your text here...

% %%%%%%%%%%%%%%%%%%%%%%%%%%%%%%%%%%%%%%%%%%%%%%%%%%%%%%%%%%%%%%%%%%%%%%%%%%%%%%%%%%%
% \section{Summary of contributions and achievements} %  use this section 
% \label{sec:intro_sum_results} % label of summary of results
% Describe clearly what you have done/created/achieved and what the major results and their implications are. 


% %%%%%%%%%%%%%%%%%%%%%%%%%%%%%%%%%%%%%%%%%%%%%%%%%%%%%%%%%%%%%%%%%%%%%%%%%%%%%%%%%%%
% \section{Organization of the report} %  use this section
% \label{sec:intro_org} % label of Org section
% Describe the outline of the rest of the report here. Let the reader know what to expect ahead in the report. Describe how you have organized your report. 

% \textbf{Example: how to refer a chapter, section, subsection}. This report is organised into seven chapters. Chapter~\ref{ch:lit_rev} details the literature review of this project. In Section~\ref{ch:method}...  % and so on.

% \textbf{Note:}  Take care of the word like ``Chapter,'' ``Section,'' ``Figure'' etc. before the \LaTeX command \textbackslash ref\{\}. Otherwise, a  sentence will be confusing. For example, In \ref{ch:lit_rev} literature review is described. In this sentence, the word ``Chapter'' is missing. Therefore, a reader would not know whether 2 is for a Chapter or a Section or a Figure.
