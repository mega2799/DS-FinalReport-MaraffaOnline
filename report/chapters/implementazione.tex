\chapter{Implementazione}
\label{ch:implementazione} % This how you label a chapter and the key (e.g., ch:into) will be used to refer this chapter ``Introduction'' later in the report. 
% the key ``ch:into'' can be used with command \ref{ch:intor} to refere this Chapter.
parla di decomposizione delle funzionalità in microservizi
dipendenze minime tra i servizi
reattività agli eventi
\section{Tecnologie}
(swagger, postman, autogenerazione javadoc?, rabbitMQ, JWT per autenticazione utenti,
 docker, nodejs, typescript, mongodb, jest, angular, figma, ...)
\section{Parte di distribuiti che non so dove mettere, sta nell'impl? p.s. dargli un nome decente}
FAULT DETECTION (con node nel cluster, nel nostro caso?)/TOLLERANCE
SCALABILITà?
verificare la scalabilità orizzontale (aumento richieste in un periodo)?
PERSISTENZA
CONSISTENZA
AFFIDABILITà
FAULTS (GIOCATORE SI DISCONNETTE NEL MEZZO DELLA PARTITA)
IL SISTEMA SI BLOCCA IN CASO DI ERRORI?
**GUARDA L'ESEMPIO DI POLITICHE AUTOVALUTAZIONE NELL'ANALISI**

monitoraggio per osservare stato attuale sistema (mi sa che noi non l'abbiamo)
\section{Screen}
\section{Optional(Mockup)}
% %%%%%%%%%%%%%%%%%%%%%%%%%%%%%%%%%%%%%%%%%%%%%%%%%%%%%%%%%%%%%%%%%%%%%%%%%%%%%%%%%%%
% \section{Problem statement}
% \label{sec:intro_prob_art}
% This section describes the investigated problem in detail. You can also have a separate chapter on ``Problem articulation.''  For some projects, you may have a section like ``Research question(s)'' or ``Research Hypothesis'' instead of a section on ``Problem statement.'

% %%%%%%%%%%%%%%%%%%%%%%%%%%%%%%%%%%%%%%%%%%%%%%%%%%%%%%%%%%%%%%%%%%%%%%%%%%%%%%%%%%%
% \section{Aims and objectives}
% \label{sec:intro_aims_obj}
% Describe the ``aims and objectives'' of your project. 

% \textbf{Aims:} The aims tell a read what you want/hope to achieve at the end of the project. The  aims define your intent/purpose in general terms.  

% \textbf{Objectives:} The objectives are a set of tasks you would perform in order to achieve the defined aims. The objective statements have to be specific and measurable through the results and outcome of the project.


% %%%%%%%%%%%%%%%%%%%%%%%%%%%%%%%%%%%%%%%%%%%%%%%%%%%%%%%%%%%%%%%%%%%%%%%%%%%%%%%%%%%
% \section{Solution approach}
% \label{sec:intro_sol} % label of Org section
% Briefly describe the solution approach and the methodology applied in solving the set aims and objectives.

% Depending on the project, you may like to alter the ``heading'' of this section. Check with you supervisor. Also, check what subsection or any other section that can be added in or removed from this template.

% \subsection{A subsection 1}
% \label{sec:intro_some_sub1}
% You may or may not need subsections here. Depending on your project's needs, add two or more subsection(s). A section takes at least two subsections. 

% \subsection{A subsection 2}
% \label{sec:intro_some_sub2}
% Depending on your project's needs, add more section(s) and subsection(s).

% \subsubsection{A subsection 1 of a subsection}
% \label{sec:intro_some_subsub1}
% The command \textbackslash subsubsection\{\} creates a paragraph heading in \LaTeX.

% \subsubsection{A subsection 2 of a subsection}
% \label{sec:intro_some_subsub2}
% Write your text here...

% %%%%%%%%%%%%%%%%%%%%%%%%%%%%%%%%%%%%%%%%%%%%%%%%%%%%%%%%%%%%%%%%%%%%%%%%%%%%%%%%%%%
% \section{Summary of contributions and achievements} %  use this section 
% \label{sec:intro_sum_results} % label of summary of results
% Describe clearly what you have done/created/achieved and what the major results and their implications are. 


% %%%%%%%%%%%%%%%%%%%%%%%%%%%%%%%%%%%%%%%%%%%%%%%%%%%%%%%%%%%%%%%%%%%%%%%%%%%%%%%%%%%
% \section{Organization of the report} %  use this section
% \label{sec:intro_org} % label of Org section
% Describe the outline of the rest of the report here. Let the reader know what to expect ahead in the report. Describe how you have organized your report. 

% \textbf{Example: how to refer a chapter, section, subsection}. This report is organised into seven chapters. Chapter~\ref{ch:lit_rev} details the literature review of this project. In Section~\ref{ch:method}...  % and so on.

% \textbf{Note:}  Take care of the word like ``Chapter,'' ``Section,'' ``Figure'' etc. before the \LaTeX command \textbackslash ref\{\}. Otherwise, a  sentence will be confusing. For example, In \ref{ch:lit_rev} literature review is described. In this sentence, the word ``Chapter'' is missing. Therefore, a reader would not know whether 2 is for a Chapter or a Section or a Figure.
