\chapter{Conclusions and Future Work}
\label{ch:con}
\section{Conclusioni}
L'architettura basata su micro-servizi si è rivelata una scelta altamente vantaggiosa per il progetto,
 garantendo modularità, scalabilità e resilienza. Ogni micro-servizio è stato progettato per gestire specifiche
  funzionalità, e la loro integrazione avviene tramite API REST e websocket orchestrate da un middleware centrale, che facilita
   una comunicazione efficiente e ben strutturata tra le componenti del sistema. L'adozione di Docker ha semplificato
    ulteriormente il processo di deploy, offrendo un ambiente di esecuzione coerente e adattabile, capace di gestire le
     diverse esigenze di configurazione e scaling e affidabilità.
% \ref{ch:results} 

L'architettura basata su micro-servizi si è rivelata una scelta altamente vantaggiosa per il progetto,
 garantendo modularità, scalabilità e resilienza. 
  Questo approccio consente una maggiore flessibilità nello sviluppo,
permettendo ai team di lavorare in modo indipendente su componenti diverse, riducendo al contempo le interdipendenze
 e i conflitti di integrazione.

Quest'integrazione avviene tramite API REST e websocket orchestrate da un middleware centrale, che facilita una 
comunicazione efficiente e ben strutturata tra le componenti del sistema. Questo sistema intermediario,
 gestisce le richieste e il bilanciamento del carico, garantendo che i dati siano trasferiti in modo rapido
 tra i vari micro-servizi. Inoltre, l'uso di websocket consente una comunicazione bidirezionale in near real-time,
  migliorando la reattività dell'applicazione.

L'adozione di Docker ha semplificato ulteriormente il processo di deploy, offrendo un ambiente di esecuzione coerente
 e adattabile. Docker, infatti, permette di incapsulare ogni micro-servizio in un container, che include tutte le
  dipendenze necessarie, eliminando problemi legati alle differenze tra ambienti di sviluppo e di test.
Questo approccio non solo facilita la distribuzione, ma migliora anche la portabilità delle applicazioni e la gestione
 delle risorse. Grazie a Docker, è possibile scalare orizzontalmente i micro-servizi in base alle necessità,
 rispondendo rapidamente alle variazioni del carico di lavoro e garantendo così un'elevata disponibilità e affidabilità del sistema.

Infine, l'architettura a micro-servizi, risulta essere altamente resiliente. In caso di fallimento di un singolo micro-servizio,
 il sistema può continuare a funzionare, con la possibilità di isolare e risolvere il problema senza impattare l'intera
  applicazione. Questo approccio modulare non solo migliora la robustezza del sistema, ma facilita anche l'implementazione
   di nuove funzionalità e l'aggiornamento di quelle esistenti, rendendo l'intero processo di sviluppo più agile e
    reattivo.