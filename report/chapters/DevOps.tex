\chapter{DevOps}
\label{ch:DevOps} % This how you label a chapter and the key (e.g., ch:into) will be used to refer this chapter ``Introduction'' later in the report. 
% the key ``ch:into'' can be used with command \ref{ch:intor} to refere this Chapter.
% \section{Git wordkflow}
\section{DVCS}

Come strumento di controllo di versione è stato scelto git, sono state adottate le seguenti best practices o git policy per garantire un flusso di lavoro coerente e prevenire conflitti di merge.
% La Git Policy rappresenta un insieme di linee guida e regole che governano l'uso di Git all'interno di un team di sviluppo. 
% Queste politiche sono fondamentali per garantire un flusso di lavoro coerente, prevenire conflitti di merge e mantenere una storia del codice chiara e facilmente navigabile. 

\subsection{Struttura dei Branch}

Una delle componenti principali della Git Policy è la gestione dei branch. La struttura tipica prevede almeno tre tipi di branch:
- \textbf{Master/Main:} Questo è il branch principale che contiene il codice di produzione. Ogni commit su questo branch dovrebbe essere stabile e pronto per il rilascio. Questo branch e stato protetto per evitare commit diretti, richiedendo quindi una revisione per ogni operazione di merge.
- \textbf{Develop:} Qui viene integrato il lavoro di sviluppo corrente. È il branch dove confluiscono le feature prima di essere preparate per il rilascio. Questo branch dovrebbe essere costantemente aggiornato e testato per assicurare che sia in uno stato pronto per la produzione.
- \textbf{Feature Branches:} Utilizzati per lo sviluppo di nuove funzionalità. Ogni feature branch deve derivare da `develop` e, una volta completata la feature, viene reintegrato in `develop` tramite una pull request.


% \subsection{Commit e Messaggi di Commit}

% La pratica adottata per i commit è stata quella di effettuare commit frequenti e significativi.
% Per facilitare la comprensione di ogni commit si è adotatto lo standard : \href{https://www.conventionalcommits.org/en/v1.0.0/}{Conventional Commits}, che divide i commit in categorie come `fix`, `feat`, `docs`, `style`, `refactor`, `test`, `chore`, `ci`, `perf`, `build`, `revert`. 

\subsection{Conventional commits}
La pratica adottata per i commit è stata quella di effettuare commit frequenti e significativi.

In ogni repo è stato adottato un sistema standard per scrivere i commit:  i conventional commit. In questo modo i commit risultano più chiari e facilmente leggibili. 
Riportiamo di seguito la nomenclatura usata:
\begin{itemize}
    \item \textbf{fix:} per i commit che risolvono un bug
    \item \textbf{feat:} per i commit che aggiungono una nuova feature
    \item \textbf{refactor:} per i commit che migliorano il codice senza aggiungere nuove funzionalità
    \item \textbf{docs:} per i commit che riguardano la documentazione
    \item \textbf{style:} per i commit che riguardano la formattazione del codice
    \item \textbf{test:} per i commit che riguardano i test
    \item \textbf{ci:} per i commit che riguardano la Continuous Integration
\end{itemize}
Inoltre sono evidenziati i breaking changes per le modifiche non più compatibili con le versioni precedenti.
La stessa nomenclatura viene utilizzata anche nel changelog.


\section{Issue template}
Il codice è open source ed è incoraggiata la collaborazione anche grazie alla presenza, in ogni repository, di template con i quali un utente può consigliare nuove feature,
segnalare un bug o suggerire un'implementazione alternativa. Sono estremamente utili per ricevere feedback dagli utenti e monitorare gli eventuali bug.

\section{Build automation}
Abbiamo deciso di implementare ogni servizio in una repository diversa. La repository fornita è un contenitore di tutte le singole repository dei servizi.
Per ogni repository è stata implementata una build automation specifica. \\
In questo modo si riesce a automatizzare i processi di build, test e deploy, ottenendo diversi vantaggi:
\begin{itemize}
    \item si riduce drasticamente il tempo per la compilazione del codice
    \item ogni volta i passaggi verranno eseguiti allo stesso modo ottenendo, quindi, riproducibilità
    \item si identificano facilmente i problemi grazie al sistema di stampe a video che si ottiene durante l'esecuzione delle GitHub Actions
    \item minimizzazione degli errori umani
    \item garantisce il continuous deployment 
\end{itemize}

\subsection{Scheletro del workflow} 

I workflow utilizzati in questo progetto, pur utilizzando actions diverse specifiche per ogni servizio e linguaggio, seguono uno schema comune che ne facilita lo sviluppo e la manutenzione.

\vspace{0.5cm}

Ogni workflow di deployment è composto da tre job principali: build, test e deploy. Seguono una logica sequenziale e non parallela, in cui la fine di ogni job decreta l'inizio di quello successivo. Il collegamento tra i job è garantito da un sistema di cache, che permette di passare i dati tra i vari job migliorandone l'efficienza e la velocità.
Questi job vengono attivati secondo alcune regole di progetto comuni alle repository, in linea con il workflow adottato. Questi ultimi non sono attivati ad ogni commit, ma solo in determinate circostanze, qui di seguito elencate:
\begin{itemize}
    \item \textbf{Pull request su develop} 
    \item \textbf{Release manuale su main pubblicata}
\end{itemize}

La struttura dei workflow è stata progettata per essere il più modulare possibile. Sono state utilizzate variabili d'ambiente per differenziare il deploy su develop e main, in modo da poter riutilizzare lo stesso workflow per entrambi i branch e produrre diverse immagini del software. Il nome dell'immagine pubblicata viene creato in modo parametrico, utilizzando il nome della repository, modificando i caratteri in minuscolo e aggiungendo la versione del software, che viene presa dal tag della release o impostata su stage o develop a seconda del branch che ha attivato il workflow.

\subsection{UserManagement e BusinessLogic} 
Essendo le due repository estremamente simili e avendo entrambe un ambiente in Node.js, è stato implementato un workflow pressoché analogo. Le actions utilizzate sono quelle di Yarn per le varie fasi di build, test e deploy, e per la creazione della cache tra i vari job.

\subsection{Middleware} 
Questa repository è stata implementata in Java, quindi il workflow è stato leggermente diverso. Per le varie fasi sono state utilizzate le action di Java e Gradle. Per il corretto funzionamento della action di setup e quindi delle successive, si è reso necessario, anche se non previsto e non consigliato, mantenere sul repository remoto il file `gradle-wrapper.jar`. Per la fase di test, invece, è stata utilizzata un'action che permette di creare un container con un database di MongoDB, in modo da poter testare il codice che interagisce con un database, il che compone una delle funzionalità principali del servizio. Anche in questo caso, password e username vengono passati tramite i secrets.

\subsection{Frontend}
La repository di frontend è stata implementata in Angular, quindi il workflow è stato leggermente diverso. Per le varie fasi sono state utilizzate le action di Yarn per le fasi di build, deploy e per la creazione della cache tra i vari job. Non è stato necessario utilizzare la fase di test, poiché il frontend di un'applicazione non ha un sistema di unit test; la comunicazione con il middleware viene testata "manualmente".




% \subsection{UserManagementMaraffa e BusinessLogic:} Essendo le due repository estremamente simili e avendo entrambe un ambiente in node.js, è stato implementato un workflow pressochè analogo.
%  Questo workflow reagisce all'evento di pull request chiuse sul branch develop e alle release, le quali vengano generate a mano. Durante lo sviluppo del workflow, infatti, si 
%  è notato come non fosse possibile catturare l'evento della creazione di un nuovo tag che avrebbe permesso di automatizzare 
%  il processo di release di github.
% e sono stati implementati 3 job: build, test e deploy.
% Dopo una checkout iniziale,
%  viene creata una cache per permettere ai job di comunicare tra loro. Successivamente alla build il sistema viene testato
%   (in entrambi i casi viene usata l'action di yarn) e infine viene effettuato il deploy del servizio containerizzato
%  Nel quale viene effettuato il chekout, si configura il nome del progetto e si configura la versione della build.
%  Se il commit viene effettuato sul main si configura il nome del tag e la versione a "LATEST", nel caso non fosse nel main viene configurata la versione a "develop".
%   Infine si effettua il login al Github container registry, la build e la push dell'immagine sul registro. L'immagine viene configurata in modo parametrico con delle variabili d'ambiente e il nome viene trasformato in minuscolo. 

% \subsection{MiddlewareMaraffa:} 
% Quando viene effettuata una release o una pull request sul branch develop ed è di tipo closed
% (non sta progredendo e quindi l’esecuzione non è in corso) il workflow del Middleware viene attivato.
% Anch'esso è costituito da tre jobs: build, test e deploy.
% Come prima cosa nella build, viene eseguito il checkout, effettuato il setup di Java e Gradle. Infine viene effettuata la 
% build di Gradle.
% Nel job del test si crea un container con un database di MongoDB. Password e username vengono passati tramite i secrets. Successivamente
% si prosegue con checkout, setup di Java utilizzando la cache per utilizzare la build di Gradle svolta precedentemente e 
% infine si esegue il test con gradlew.
% Per quanto riguarda il deploy, si effettua il checkout utilizzando un token salvato all'interno di un secret. La parte
% restante coincide con quella delle repository precedenti.
% push su main, pull request su main e develop
% build e test se pullrequest.merge = true 

% Parte finale 




\section{Continuous integration}
Per migliorare la qualità del software e velocizzare il processo di sviluppo, è stata implementata la Continuous Integration.
Questa pratica permette di integrare il codice frequentemente, evitando possibili rischi d'integrazione e consetendo 
l'esecuzione di test automatici. Grazie al sistema di notifiche, inoltre, è possibile ricevere feedback immediato sullo stato delle Github Actions,
per accertarsi che sia sempre tutto funzionante e non siano stati introdotti bug. Mantenendo il sistema continuamente monitorato,
si migliora anche l'effiicienza e la collaborazione tra i membri del team.
% -ci (GHA, deploy pages, usato due servizi di hosting github e gitlab (per copia all'intero di repo di magnani))


% %%%%%%%%%%%%%%%%%%%%%%%%%%%%%%%%%%%%%%%%%%%%%%%%%%%%%%%%%%%%%%%%%%%%%%%%%%%%%%%%%%%
% \section{Problem statement}
% \label{sec:intro_prob_art}
% This section describes the investigated problem in detail. You can also have a separate chapter on ``Problem articulation.''  For some projects, you may have a section like ``Research question(s)'' or ``Research Hypothesis'' instead of a section on ``Problem statement.'

% %%%%%%%%%%%%%%%%%%%%%%%%%%%%%%%%%%%%%%%%%%%%%%%%%%%%%%%%%%%%%%%%%%%%%%%%%%%%%%%%%%%
% \section{Aims and objectives}
% \label{sec:intro_aims_obj}
% Describe the ``aims and objectives'' of your project. 

% \textbf{Aims:} The aims tell a read what you want/hope to achieve at the end of the project. The  aims define your intent/purpose in general terms.  

% \textbf{Objectives:} The objectives are a set of tasks you would perform in order to achieve the defined aims. The objective statements have to be specific and measurable through the results and outcome of the project.


% %%%%%%%%%%%%%%%%%%%%%%%%%%%%%%%%%%%%%%%%%%%%%%%%%%%%%%%%%%%%%%%%%%%%%%%%%%%%%%%%%%%
% \section{Solution approach}
% \label{sec:intro_sol} % label of Org section
% Briefly describe the solution approach and the methodology applied in solving the set aims and objectives.

% Depending on the project, you may like to alter the ``heading'' of this section. Check with you supervisor. Also, check what subsection or any other section that can be added in or removed from this template.

% \subsection{A subsection 1}
% \label{sec:intro_some_sub1}
% You may or may not need subsections here. Depending on your project's needs, add two or more subsection(s). A section takes at least two subsections. 

% \subsection{A subsection 2}
% \label{sec:intro_some_sub2}
% Depending on your project's needs, add more section(s) and subsection(s).

% \subsubsection{A subsection 1 of a subsection}
% \label{sec:intro_some_subsub1}
% The command \textbackslash subsubsection\{\} creates a paragraph heading in \LaTeX.

% \subsubsection{A subsection 2 of a subsection}
% \label{sec:intro_some_subsub2}
% Write your text here...

% %%%%%%%%%%%%%%%%%%%%%%%%%%%%%%%%%%%%%%%%%%%%%%%%%%%%%%%%%%%%%%%%%%%%%%%%%%%%%%%%%%%
% \section{Summary of contributions and achievements} %  use this section 
% \label{sec:intro_sum_results} % label of summary of results
% Describe clearly what you have done/created/achieved and what the major results and their implications are. 


% %%%%%%%%%%%%%%%%%%%%%%%%%%%%%%%%%%%%%%%%%%%%%%%%%%%%%%%%%%%%%%%%%%%%%%%%%%%%%%%%%%%
% \section{Organization of the report} %  use this section
% \label{sec:intro_org} % label of Org section
% Describe the outline of the rest of the report here. Let the reader know what to expect ahead in the report. Describe how you have organized your report. 

% \textbf{Example: how to refer a chapter, section, subsection}. This report is organised into seven chapters. Chapter~\ref{ch:lit_rev} details the literature review of this project. In Section~\ref{ch:method}...  % and so on.

% \textbf{Note:}  Take care of the word like ``Chapter,'' ``Section,'' ``Figure'' etc. before the \LaTeX command \textbackslash ref\{\}. Otherwise, a  sentence will be confusing. For example, In \ref{ch:lit_rev} literature review is described. In this sentence, the word ``Chapter'' is missing. Therefore, a reader would not know whether 2 is for a Chapter or a Section or a Figure.
