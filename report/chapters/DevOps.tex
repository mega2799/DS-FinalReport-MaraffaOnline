\chapter{DevOps}
\label{ch:DevOps} % This how you label a chapter and the key (e.g., ch:into) will be used to refer this chapter ``Introduction'' later in the report. 
% the key ``ch:into'' can be used with command \ref{ch:intor} to refere this Chapter.
% \section{Git wordkflow}
\section{DVCS}

Come strumento di controllo di versione è stato scelto git, sono state adottate le seguenti best practices o git policy per garantire un flusso di lavoro coerente e prevenire conflitti di merge.
% La Git Policy rappresenta un insieme di linee guida e regole che governano l'uso di Git all'interno di un team di sviluppo. 
% Queste politiche sono fondamentali per garantire un flusso di lavoro coerente, prevenire conflitti di merge e mantenere una storia del codice chiara e facilmente navigabile. 

\subsection{Struttura dei Branch}

Una delle componenti principali della Git Policy è la gestione dei branch. La struttura tipica prevede almeno tre tipi di branch:
- \textbf{Master/Main:} Questo è il branch principale che contiene il codice di produzione. Ogni commit su questo branch dovrebbe essere stabile e pronto per il rilascio. Questo branch e stato protetto per evitare commit diretti, richiedendo quindi una revisione per ogni operazione di merge.
- \textbf{Develop:} Qui viene integrato il lavoro di sviluppo corrente. È il branch dove confluiscono le feature prima di essere preparate per il rilascio. Questo branch dovrebbe essere costantemente aggiornato e testato per assicurare che sia in uno stato pronto per la produzione.
- \textbf{Feature Branches:} Utilizzati per lo sviluppo di nuove funzionalità. Ogni feature branch deve derivare da `develop` e, una volta completata la feature, viene reintegrato in `develop` tramite una pull request.


% \subsection{Commit e Messaggi di Commit}

% La pratica adottata per i commit è stata quella di effettuare commit frequenti e significativi.
% Per facilitare la comprensione di ogni commit si è adotatto lo standard : \href{https://www.conventionalcommits.org/en/v1.0.0/}{Conventional Commits}, che divide i commit in categorie come `fix`, `feat`, `docs`, `style`, `refactor`, `test`, `chore`, `ci`, `perf`, `build`, `revert`. 

\subsection{Conventional commits}
La pratica adottata per i commit è stata quella di effettuare commit frequenti e significativi.

In ogni repo è stato adottato un sistema standard per scrivere i commit:  i conventional commit. In questo modo i commit risultano più chiari e facilmente leggibili. 
Riportiamo di seguito la nomenclatura usata:
\begin{itemize}
    \item \textbf{fix:} per i commit che risolvono un bug
    \item \textbf{feat:} per i commit che aggiungono una nuova feature
    \item \textbf{refactor:} per i commit che migliorano il codice senza aggiungere nuove funzionalità
    \item \textbf{docs:} per i commit che riguardano la documentazione
    \item \textbf{style:} per i commit che riguardano la formattazione del codice
    \item \textbf{test:} per i commit che riguardano i test
    \item \textbf{ci:} per i commit che riguardano la Continuous Integration
\end{itemize}
Inoltre sono evidenziati i breaking changes per le modifiche non più compatibili con le versioni precedenti.
La stessa nomenclatura viene utilizzata anche nel changelog.


\section{Issue template}
-per discutere l'impl di una nuova idea, sottomettere feature proposal, reporting bugs e malfunzionamenti, elaborare nuove impl
[il tip above non l'ho inserito nella relazione di SPE]

\section{Continuous integration}
-ci (GHA, deploy pages, usato due servizi di hosting github e gitlab (per copia all'intero di repo di magnani))

\section{Build automation}

% %%%%%%%%%%%%%%%%%%%%%%%%%%%%%%%%%%%%%%%%%%%%%%%%%%%%%%%%%%%%%%%%%%%%%%%%%%%%%%%%%%%
% \section{Problem statement}
% \label{sec:intro_prob_art}
% This section describes the investigated problem in detail. You can also have a separate chapter on ``Problem articulation.''  For some projects, you may have a section like ``Research question(s)'' or ``Research Hypothesis'' instead of a section on ``Problem statement.'

% %%%%%%%%%%%%%%%%%%%%%%%%%%%%%%%%%%%%%%%%%%%%%%%%%%%%%%%%%%%%%%%%%%%%%%%%%%%%%%%%%%%
% \section{Aims and objectives}
% \label{sec:intro_aims_obj}
% Describe the ``aims and objectives'' of your project. 

% \textbf{Aims:} The aims tell a read what you want/hope to achieve at the end of the project. The  aims define your intent/purpose in general terms.  

% \textbf{Objectives:} The objectives are a set of tasks you would perform in order to achieve the defined aims. The objective statements have to be specific and measurable through the results and outcome of the project.


% %%%%%%%%%%%%%%%%%%%%%%%%%%%%%%%%%%%%%%%%%%%%%%%%%%%%%%%%%%%%%%%%%%%%%%%%%%%%%%%%%%%
% \section{Solution approach}
% \label{sec:intro_sol} % label of Org section
% Briefly describe the solution approach and the methodology applied in solving the set aims and objectives.

% Depending on the project, you may like to alter the ``heading'' of this section. Check with you supervisor. Also, check what subsection or any other section that can be added in or removed from this template.

% \subsection{A subsection 1}
% \label{sec:intro_some_sub1}
% You may or may not need subsections here. Depending on your project's needs, add two or more subsection(s). A section takes at least two subsections. 

% \subsection{A subsection 2}
% \label{sec:intro_some_sub2}
% Depending on your project's needs, add more section(s) and subsection(s).

% \subsubsection{A subsection 1 of a subsection}
% \label{sec:intro_some_subsub1}
% The command \textbackslash subsubsection\{\} creates a paragraph heading in \LaTeX.

% \subsubsection{A subsection 2 of a subsection}
% \label{sec:intro_some_subsub2}
% Write your text here...

% %%%%%%%%%%%%%%%%%%%%%%%%%%%%%%%%%%%%%%%%%%%%%%%%%%%%%%%%%%%%%%%%%%%%%%%%%%%%%%%%%%%
% \section{Summary of contributions and achievements} %  use this section 
% \label{sec:intro_sum_results} % label of summary of results
% Describe clearly what you have done/created/achieved and what the major results and their implications are. 


% %%%%%%%%%%%%%%%%%%%%%%%%%%%%%%%%%%%%%%%%%%%%%%%%%%%%%%%%%%%%%%%%%%%%%%%%%%%%%%%%%%%
% \section{Organization of the report} %  use this section
% \label{sec:intro_org} % label of Org section
% Describe the outline of the rest of the report here. Let the reader know what to expect ahead in the report. Describe how you have organized your report. 

% \textbf{Example: how to refer a chapter, section, subsection}. This report is organised into seven chapters. Chapter~\ref{ch:lit_rev} details the literature review of this project. In Section~\ref{ch:method}...  % and so on.

% \textbf{Note:}  Take care of the word like ``Chapter,'' ``Section,'' ``Figure'' etc. before the \LaTeX command \textbackslash ref\{\}. Otherwise, a  sentence will be confusing. For example, In \ref{ch:lit_rev} literature review is described. In this sentence, the word ``Chapter'' is missing. Therefore, a reader would not know whether 2 is for a Chapter or a Section or a Figure.
