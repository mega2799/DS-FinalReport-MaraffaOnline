\chapter{Analisi}
\label{ch:analisi} %Label of the chapter lit rev. The key ``ch:lit_rev'' can be used with command \ref{ch:lit_rev} to refer this Chapter.
\section{Obiettivi}
I principali requisiti funzionali e obiettivi del sistema sono:

\begin{itemize}
    \item \textbf{Costruzione di un'architettura a micro-servizi}.

    \item \textbf{Autenticazione e registrazione utenti: } possibilità di accedere ad un'area privata per poter visualizzare informazioni personali, oltre alla possibilità di poter giocare senza un proprio account in modalità 'ospite'.
    \item \textbf{Creazione e gestione partita: } gli utenti possono creare o partecipare ad una partita che inizierà soltanto quando 4 giocatori saranno presenti e disposti in due squadre da 2.

    \item \textbf{Chat di gioco: } possibilità di scambiare messaggi.

    \item \textbf{Nuova modalità di gioco} in cui se si commette un errore di gioco la partita viene conclusa conferendo alla squadra avversaria il massimo dei punti.

    \item \textbf{Persistenza dei dati: } salvataggio su database degli utenti e delle statistiche delle partite (mani giocate, briscola, etc...).
\end{itemize}
\section{Politiche di autovalutazione} 
\begin{itemize}
    \item l'elenco delle partite iniziate/finite deve rimanere aggiornato
    \item se un giocatore si disconnette la partita termina per tutti (analizzando i casi pratici si è notato che quando un giocatore si disconnette è perchè non ha intenzione di rientrare, è pertanto inutile far attendere gli altri giocatori per una possibile riconnessione)
    \item nel caso di connessione lenta o malfunzionamenti da parte di un giocatore, il giocatore potrà rimanere sempre aggiornato sullo stato corrente della partita riaggiornado la pagina
    \item quando un giocatore clicca sul tasto "esci", verrà comunicato a tutti gli altri giocatori che la partita è terminata e verranno reindirizzati alla schermata principale
    \item nel caso di un crash di un servizio si deve garantire che il servizio venga riavviato in automatico evitando che il sistema si blocchi
    % \item se crasha un servizio docker ha già una funzione builtin di riavviare in automatico il container
\end{itemize}
% Si devono seguire le regole per completare una sessione di gioco.
% • Se un giocatore che non `e l’ultimo rimasto si disconnette si disconnette la
% partita deve continuare. Nel caso in cui la partita fosse ancora in e il giocatre disconneso si volesse riconnettere il server far`a in modo di mostrare
% al giocatre le carte che aveva al momento della disconnessione
% • Se l’ultimo giocatore si disconnette allora la partita deve terminare.
% • L’elenco delle partite aperte/chiuse deve rimanere aggiornato. Inoltre,
% devono essere costantemente sincronizzati i client dei giocatori all’interno
% della stessa partita, in modo da garantire la coerenza del gioco.
% • L’obiettivo principale del software sviluppato `e garantire una solida capacit`a di adattamento alle dimensioni, una fluidit`a accettabile nell’esperienza
% di gioco e una sufficiente resilienza agli errori. In modo specifico, il servizio web necessita di gestire in modo efficace un grande numero di utenti
% simultaneamente, mentre l’esecuzione del gioco deve risultare scorrevole
% sul dispositivo dell’utente, evitando inconvenienti legati alle prestazioni sia
% grafiche che funzionali. Inoltre, `e essenziale che il sistema non si blocchi
% in caso di eventuali problemi.

\section{Requisiti e casi d'uso}
\subsection{Requisiti}
\begin{enumerate}
    \item Account
        \begin{enumerate}
            \item Login
            \item Registrazione
            \item Recupero password
            \item Visualizzazione profilo
            \item Modifica password
            \item Possibilità di scegliere se giocare come ospite o effettuare il login
        \end{enumerate}
    \item Realizzazione partita
        \begin{enumerate}
            \item Creazione partita
            \item Partecipazione partita
            \item gioca carta
            \item Inizio partita
            \item Fine mano
            \item Fine partita
        \end{enumerate}
    \item Chat di gioco
        \begin{enumerate}
            \item chat globale
            \item chat partita
        \end{enumerate}
    \item Possibilità di scegliere un compagno di squadra
    \item Scelta del seme, parole consentite
    \item Modalità di gioco 11 a 0
    \item Gestione punteggio
        \begin{enumerate}
            \item Calcolo totale e parziale (Gestione per ogni mano) del punteggio
            \item Maraffa/Cricca (+3 punti)
        \end{enumerate}
    \item Servizio gestione utenti
    \item Salvataggio statistiche
    \item Realizzazione GUI
        \begin{enumerate}
            \item Refactor della GUI esistente
            \item Rinnovamento GUI
        \end{enumerate}
\end{enumerate}
\subsection{Casi d'uso}
Si riporta di seguito lo schema dei casi d'uso che modella l'interazione dell'utente con l'applicazione.
\begin{figure}[h!]
    \centering 
    \includegraphics[scale=0.45]{report/img/Casi_duso.png}
    \caption{Casi d'uso}
    \label{use_case}
\end{figure}

% Take a note of the commands \textbackslash cite\{\} and \textbackslash citep\{\}. The command \textbackslash cite\{\} will write like ``Author et al. (2019)'' style for Harvard, APA and Chicago style. The command \textbackslash citep\{\} will write like ``(Author et al., 2019).'' Depending on how you construct a sentence, you need to use them smartly. Check the examples of \textbf{in-text citation} of sources listed here [This template recommends the \textbf{Harvard style} of referencing.]:
% \begin{itemize}
%     \item \cite{lamport1994latex} has written a comprehensive guide on writing in \LaTeX ~[Example of \textbackslash cite\{\} ].
%     \item If \LaTeX~is used efficiently and effectively, it helps in writing a very high-quality project report~\citep{lamport1994latex} ~[Example of \textbackslash citep\{\} ].   
%     \item A detailed APA, Harvard, and Chicago referencing style guide are available in~\citep{uor_refernce_style}.
% \end{itemize}




%\noindent
%\textbf{\color{red}MUST}: do read the university guidelines on the definition of plagiarism as well as the guidelines on how to avoid plagiarism~\citep{uor_plagiarism}.





