\chapter{Introduzione}
\label{ch:into} % This how you label a chapter and the key (e.g., ch:into) will be used to refer this chapter ``Introduction'' later in the report. 
% the key ``ch:into'' can be used with command \ref{ch:intor} to refere this Chapter.
MaraffaOnline è un'applicazione che permette alle persone di giocare al gioco di carte \href{https://it.wikipedia.org/wiki/Marafone_Beccacino}{Maraffa/Beccacino}. 
Il progetto consiste nell’eseguire una manutenzione evolutiva del gioco di carte reperibile su MaraffaOnline, attualmente sviluppato dalla prof.ssa Lumini.
In particolare la nuova versione avrà un'architettura a microservizi e introdurrà anche nuove funzionalità come formazione personalizzata delle squadre, una nuova modalità 
di gioco (vittoria 11 a 0 in caso di violazione delle regole da parte di una squadra), salvataggio delle statistiche delle partite e degli utenti, ...
\\
Per realizzare questa evoluzione, il progetto sarà suddiviso in diverse fasi, ognuna delle quali si concentrerà su un aspetto specifico dell'architettura
e delle funzionalità. 

La prima fase riguarderà la decomposizione dell'applicazione monolitica esistente in microservizi indipendenti, che comunicheranno
tra loro attraverso API RESTful. 

Nella seconda fase, saranno sviluppati i sistemi di salvataggio delle statistiche e la chat. Le statistiche di gioco saranno memorizzate in un database relazionale,
permettendo agli utenti di visualizzare le loro performance nel tempo e confrontarsi con altri giocatori.
Il gioco disporrà di due chat: una globale, accessibile a tutti gli utenti, e una relativa alla partita, accessibile solo ai giocatori della partita in corso.

La terza fase si concentrerà sull'implementazione delle nuove funzionalità, partendo dalla formazione personalizzata delle squadre fino alla modalità 11 a 0, la quale assicurerà
che le violazioni delle regole siano rilevate automaticamente.

Il progetto sarà gestito utilizzando metodologie agili, con sprint di sviluppo e feedback continuo da parte degli utenti.
In questo modo, sarà possibile adattare rapidamente il piano di lavoro in base ai consigli,
garantendo al contempo un prodotto finale di alta qualità e in linea con le aspettative degli utenti.