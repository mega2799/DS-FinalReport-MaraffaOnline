\chapter{Interazioni}
\label{ch:interazioni} % This how you label a chapter and the key (e.g., ch:into) will be used to refer this chapter ``Introduction'' later in the report. 
% the key ``ch:into'' can be used with command \ref{ch:intor} to refere this Chapter.
% interazioni tra i componenti, come comunicano, diagrammi d'attività.
% (parlare di api gateway (permette comunicazione tra client esterni con i microservizi, redireziona le richieste ai servizi))

\section{Frontend $ \leftrightarrow $ WebSocket} 

Come precedentemente citato, il middleware è stato implementato con WebSocket per poter comunicare in modo reattivo con il frontend. Queste comunicazioni frontend <-> middleware sono in buona parte scambi di messaggi tramite HTTP, ma in alcuni casi è necessario un canale di comunicazione bidirezionale. Uno degli esempi più "eclatanti" è la dashboard, in cui, in una prima implementazione, era stato utilizzato un polling per aggiornare i dati riguardanti le partite in corso per permettere agli utenti di partecipare. Immaginando uno scenario distribuito in cui ci sono N utenti contemporaneamente connessi al sistema, quella soluzione avrebbe sovraccaricato il sistema. Invece, tramite il canale bidirezionale delle WebSocket, è direttamente il middleware, che al proprio interno ha un servizio che mantiene una mappa dei client connessi, a notificare tramite un'operazione di broadcast tutti i client della creazione di una nuova partita.

\subsection{Chat globale e privata}

Un grande vantaggio dell'uso delle WebSocket sta nella facile implementazione delle chat. Questo sistema richiedeva l'impiego di due diverse chat per permettere la comunicazione tra utenti sia globalmente che all'interno di una singola partita. 
Anche in questo caso, il middleware si occupa di instradare i messaggi alle giuste chat, garantendo la privacy delle conversazioni. Il frontend, invece, si occupa di visualizzare i messaggi e di inviarli al middleware. La gestione della persistenza delle chat è stata implementata tramite l'uso del session storage del browser all'interno del middleware, permettendo di mantenere la chat anche in caso di refresh della pagina.

\subsection{Reattività e aggiornamento in tempo reale}
Tramite l'utilizzo delle WebSocket, è possibile ottenere un aggiornamento in tempo reale delle informazioni, senza dover fare polling continuo al server. Questa tecnologia è stata ampiamente utilizzata nella pagina della partita, permettendo ai client di ricevere aggiornamenti immediati, ad esempio sulle carte giocate da altri giocatori o su eventi scatenati da azioni specifiche, come l'abbandono di un giocatore, la gestione dei punteggi o la conclusione della partita stessa.

\section{Stack Docker $ \rightarrow $ reti interne e IP privati}

Il sistema è stato progettato per essere eseguito su qualsiasi piattaforma e si è arrivati a creare uno stack completo di Docker Compose. Questo ha permesso anche la creazione di reti interne tra i container che vengono utilizzate per la comunicazione tra i servizi. È stato possibile assegnare IP privati ai container, permettendo di non esporre i servizi all'esterno e di mantenere un'architettura più sicura. Tra le variabili d'ambiente che vengono passate ai container c'è anche l'indirizzo del servizio a cui connettersi, e questo indirizzo è il nome del servizio all'interno della rete. Questo permette di non dover modificare il codice ogni volta che si cambia l'indirizzo del servizio a cui connettersi, ma basta modificare il file di configurazione del Docker Compose.
\subsection{Nginx} \label{Nginx}
Per quanto riguarda il frontend, è stato utilizzato Nginx per eseguire l'applicazione all'interno del container e per poterla raggiungere dall'esterno. Per fare ciò è stato necessario configurare Nginx tramite un file di configurazione che ha permesso anche di gestire un'operazione di reverse proxy per indirizzare le chiamate al backend.

\begin{lstlisting}[language=Python, caption={Configurazione Nginx del front-end}, label=list:nginx_frontend]
server {
    listen 80;
    
    location / {
        root /usr/share/nginx/html;
        index index.html index.htm;
        try_files $uri $uri/ /index.html;
    }

    location /api/ {
        proxy_pass "http://${API_HOST}:${API_PORT}/";
        proxy_http_version 1.1;
        proxy_set_header Upgrade $http_upgrade;
        proxy_set_header Connection 'upgrade';
        proxy_set_header Host $host;
        proxy_cache_bypass $http_upgrade;
    }
}
\end{lstlisting}
La scelta di Nginx è stata fatta soprattutto per la sicurezza nel "mascherare" le chiamate fatte al backend, in modo da non esporre l'indirizzo del servizio all'esterno e poter superare eventuali problemi di CORS. L'indirizzo del backend è stato configurato anch'esso tramite variabili d'ambiente, in modo da poterlo cambiare facilmente in base all'ambiente in cui si trova il container, e queste variabili vengono sostituite a seconda dell'ambiente di deploy in cui l'applicativo Angular viene eseguito.

% %%%%%%%%%%%%%%%%%%%%%%%%%%%%%%%%%%%%%%%%%%%%%%%%%%%%%%%%%%%%%%%%%%%%%%%%%%%%%%%%%%%
% \section{Problem statement}
% \label{sec:intro_prob_art}
% This section describes the investigated problem in detail. You can also have a separate chapter on ``Problem articulation.''  For some projects, you may have a section like ``Research question(s)'' or ``Research Hypothesis'' instead of a section on ``Problem statement.'

% %%%%%%%%%%%%%%%%%%%%%%%%%%%%%%%%%%%%%%%%%%%%%%%%%%%%%%%%%%%%%%%%%%%%%%%%%%%%%%%%%%%
% \section{Aims and objectives}
% \label{sec:intro_aims_obj}
% Describe the ``aims and objectives'' of your project. 

% \textbf{Aims:} The aims tell a read what you want/hope to achieve at the end of the project. The  aims define your intent/purpose in general terms.  

% \textbf{Objectives:} The objectives are a set of tasks you would perform in order to achieve the defined aims. The objective statements have to be specific and measurable through the results and outcome of the project.


% %%%%%%%%%%%%%%%%%%%%%%%%%%%%%%%%%%%%%%%%%%%%%%%%%%%%%%%%%%%%%%%%%%%%%%%%%%%%%%%%%%%
% \section{Solution approach}
% \label{sec:intro_sol} % label of Org section
% Briefly describe the solution approach and the methodology applied in solving the set aims and objectives.

% Depending on the project, you may like to alter the ``heading'' of this section. Check with you supervisor. Also, check what subsection or any other section that can be added in or removed from this template.

% \subsection{A subsection 1}
% \label{sec:intro_some_sub1}
% You may or may not need subsections here. Depending on your project's needs, add two or more subsection(s). A section takes at least two subsections. 

% \subsection{A subsection 2}
% \label{sec:intro_some_sub2}
% Depending on your project's needs, add more section(s) and subsection(s).

% \subsubsection{A subsection 1 of a subsection}
% \label{sec:intro_some_subsub1}
% The command \textbackslash subsubsection\{\} creates a paragraph heading in \LaTeX.

% \subsubsection{A subsection 2 of a subsection}
% \label{sec:intro_some_subsub2}
% Write your text here...

% %%%%%%%%%%%%%%%%%%%%%%%%%%%%%%%%%%%%%%%%%%%%%%%%%%%%%%%%%%%%%%%%%%%%%%%%%%%%%%%%%%%
% \section{Summary of contributions and achievements} %  use this section 
% \label{sec:intro_sum_results} % label of summary of results
% Describe clearly what you have done/created/achieved and what the major results and their implications are. 


% %%%%%%%%%%%%%%%%%%%%%%%%%%%%%%%%%%%%%%%%%%%%%%%%%%%%%%%%%%%%%%%%%%%%%%%%%%%%%%%%%%%
% \section{Organization of the report} %  use this section
% \label{sec:intro_org} % label of Org section
% Describe the outline of the rest of the report here. Let the reader know what to expect ahead in the report. Describe how you have organized your report. 

% \textbf{Example: how to refer a chapter, section, subsection}. This report is organised into seven chapters. Chapter~\ref{ch:lit_rev} details the literature review of this project. In Section~\ref{ch:method}...  % and so on.

% \textbf{Note:}  Take care of the word like ``Chapter,'' ``Section,'' ``Figure'' etc. before the \LaTeX command \textbackslash ref\{\}. Otherwise, a  sentence will be confusing. For example, In \ref{ch:lit_rev} literature review is described. In this sentence, the word ``Chapter'' is missing. Therefore, a reader would not know whether 2 is for a Chapter or a Section or a Figure.
